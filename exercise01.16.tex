\textbf{Exercise 1.16}: Let's formalize the setting. Our sample space $\Omega$
consists of all possible sequences of rolls. Furthermore, let $E_1$ be the event
that all three dice show the same number on the first roll. Let $E_2$ be the event
that exactly two of the three dice show the same number on the first roll, and let
$E_3$ be the event that all three dice show different numbers on the first roll.
Obviously $E_1, E_2$ and $E_3$ are mutually disjoint events such that
$E_1 \cup E_2 \cup \E_3 = \Omega$. Additionally, let $W$ be the event that the
player wins the game.
\begin{enumerate}
  \item[(a)] The probability that all three dice show a specified number is
    $\frac{1}{6^3}$. Since there are six different numbers and it does not matter
    which of them the dice show as long as they all show the same number,
    \[ \pr(E_1) = 6 \cdot \frac{1}{6^3} = \frac{1}{36}.\]

  \item[(b)] The probability that two dice show a specified number and one die
    shows any other number is $\frac{1}{6^2} \cdot \frac{5}{6} = \frac{5}{6^3}$.
    Since there are six different numbers and it does not matter which two of the
    three dice show the same number,
    \[ \pr(E_2) = \binom{3}{2} \cdot 6 \cdot \frac{5}{6^3} = \frac{5}{12}.\]

  \item[(c)] Assume that exactly two of the three dice show the same number on
    the first roll. According to the player's strategy, the player re-rolls the
    die that does not match. The probability that she wins after the second roll
    is therefore $\frac{1}{6}$. Assume that she does not win after the second roll.
    Then again, she re-rolls the die that does not match and the probability that
    the die shows the requested number is $\frac{1}{6}$. Therefore, for the
    probability that the player wins, conditioned on exactly two of the three dice
    showing the same number on the first roll holds
    \[ \pr(W|E_2) =\frac{1}{6} + \frac{5}{6} \cdot \frac{1}{6} = \frac{11}{36}. \]

  \item[(d)] Let's first determine the probability that all dice show different
    numbers on the first roll. This happens when the second die shows a different
    number than the first one, and the third die shows a different number than the
    first and the sceond die. Therefore $\pr(E_3) = \frac{5}{6} \cdot \frac{4}{6}
    = \frac{5}{9}$. \\
    Furthermore, let us consider the probability that the player wins, conditioned
    on all three dice showing different numbers on the first roll. According to the
    player's strategy, the player re-rolls all dice. Since the rolls are independent,
    the probability that she wins after the second roll is $\frac{1}{36}$. The
    probability that all dice are different and she wins after the third roll is
    $\frac{5}{9} \cdot \frac{1}{36}$. The probability that exactly two dice show the
    same number and she wins after the third roll is $\frac{5}{12} \cdot \frac{1}{6}$.  Altogether, one has
    \[ \pr(W|E_3)
        = \frac{1}{36} + \frac{5}{9} \cdot \frac{1}{36} +
          \frac{5}{12} \cdot \frac{1}{6}
        = \frac{73}{648}.
    \]
    We can now apply the Law of total probability to compute the probability that
    the player wins the game.
    \[
      \pr(W)
        = \sum_{i=1}^3 \pr(W|E_i) \cdot \pr(E_i)
        = 1 \cdot \frac{1}{36} + \frac{11}{36} \cdot \frac{5}{12} + \frac{73}{648} \cdot \frac{5}{9}
        = \frac{2,539}{11,664}
        \approx 0.218.
    \]
\end{enumerate}
