\paragraph{Exercise 7.5} $~$ \\[0.2cm]
\textbf{Theorem 7.19}: Let $i$ be a state. Then
\begin{enumerate}
  \item[(i)] state $i$ is recurrent if and only if
    $\sum_{n=1}^{\infty} P_{i,i}^n = \infty$, and
  \item[(ii)] state $i$ is transient if and only if
    $\sum_{n=1}^{\infty} P_{i,i}^n < \infty$.
\end{enumerate}
\textit{Proof}. Let $i$ be a state of a Markov chain. We want to compute, in two
different ways, the expected number of visits to $i$ conditioned that we are
starting from state $i$, that is $E\left(\text{number of visits to }i | X_0 =
i\right)$. \\
Let $f_i \in [0,1]$, so that $f_i = \sum_{t\geq1}r_{i,i}^t$. Furthermore, let
$\mathbb{I}_n$ be an indicator random variable, so that
\[
    \mathbb{I}_n = \begin{cases}
      1,  &\text{if the process is at state }i \text{ at time }n; \\
      0,  &\text{else}.
  \end{cases}
\]
Hence, $\pr\left(\mathbb{I}_n = 1\right) = P_{i,i}^n$. Consider the random variable
\[ V_i := \sum_{n=1}^{\infty} \mathbb{I}_n,  \]
which counts the number of returns of the Markov chain to state $i$. On the one
hand, applying Lemma 2.9, the expectation of $V_i$ can be computed as follows
\begin{align}
  \E[V_i]
    = \sum_{k=1}^{\infty} \pr\left(V_i \geq k \right)
    = \sum_{k=1}^{\infty} f_i^k.
\end{align}
On the other hand, applying the linearity of expectation (Theorem 2.1),
\begin{align}
  \E[V_i]
    = \E\left[\sum_{n=1}^{\infty} \mathbb{I}_n\right]
    = \sum_{n=1}^{\infty}\E[\mathbb{I}_n]
    = \sum_{n=1}^{\infty}P_{i,i}^n.
\end{align}
We will now consider two different cases. \\
Assume that state $i$ is recurrent. Then $f_i = \sum_{t\geq1}r_{i,i}^t = 1$ by
definition of a recurrent state. It follows that $E[V_i] = \infty$ by
equation (7.1). Hence, $\sum_{n=1}^{\infty}P_{i,i}^n = \infty$ by (7.2). \\
Assume that state $i$ is transient. Then, $f_i < 1$. Note that $\sum_{k=1}^{\infty}
f_i^k$ is a geometric series. Since $f_i \not= 1$, $\sum_{k=1}^{\infty} f_i^k$
converges, that is $E[V_i] < \infty$. Hence, $\sum_{n=1}^{\infty}P_{i,i}^n <
\infty$. \\[0.3cm]
\textbf{Theorem 7.20}: If one state in a communicating class is transient
(respectively, recurrent) then all states in that class are transient
(respectively, recurrent). \\[0.2cm]
\textit{Proof}. Let $i$ be a transient state and let $j \leftrightarrow i$.
Since $i$ and $j$ are communicating, there exist $l,m \in \mathbb{N}$ such that
$P_{i,j}^l > 0$ and $P_{j,i}^m > 0$. For all $n \in \mathbb{N}$ holds that
\[ P_{i,j}^l P_{j,j}^n P_{j,i}^m \leq P_{i,i}^{l+n+m}. \]
Consequently,
\[
  \sum_{n=1}^{\infty} P_{j,j}^n
    \leq \frac{1}{P_{i,j}^l P_{j,i}^m} \sum_{n=1}^{\infty} P_{i,i}^{l+n+m}
    \leq \frac{1}{P_{i,j}^l P_{j,i}^m} \sum_{n=1}^{\infty} P_{i,i}^n
    < \infty,
\]
where the last inequality uses Theorem 7.19 and our assumption that $i$ is
transient. Since $\sum_{n=1}^{\infty} P_{j,j}^n < \infty$, $j$ is also a transient
state. Hence, if one state in a communicating class is transient then all states
in that class are transient. This is equivalent to the statement that if one
state in a communicating class is recurrent then all states in that class are
recurrent.
