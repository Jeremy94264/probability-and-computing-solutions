\textbf{Exercise 1.3}: We shuffle a standard deck of cards, obtaining a permutation
that is uniform over all 52! possible permutations.

\begin{enumerate}
  \item[(a)] We are interested in the probability of the event $E_a$ that the
    first two cards include at least one ace. Therefore we compute the complementary
    probability of $E_a$, that is the probability $\pr\left(\bar{E_a}\right)$
    that the first two cards inlude no ace. Since a standard deck of cards
    includes 4 aces,
    \[ \pr\left(\bar{E_a}\right) = \frac{48}{52} \cdot \frac{47}{51} = \frac{188}{221}.\]
    Consequently,
    \[ \pr\left(E_a\right) = 1 - \pr\left(\bar{E_a}\right) = 1 - \frac{188}{221} = \frac{33}{221}.\]

  \item[(b)] Let $E_b$ be the event that the first five cards include at least one
    ace. Again we compute the complementary probability of the specified event.
    \[
      \pr\left(E_b\right)
        = 1 - \pr\left(\bar{E_b}\right)
        = 1 - \frac{48}{52} \cdot \frac{47}{51} \cdot \frac{46}{50} \cdot \frac{45}{49} \cdot \frac{44}{48}
        = 1 - \frac{205,476,480}{311,875,200}.
    \]
    Alternatively, we could argue that the described situation suites the conditions
    for a hypergeometric situation, so that
    \[
      \pr\left(E_b\right)
        = 1 - \frac{\binom{4}{0} \cdot \binom{48}{5}}{\binom{52}{5}}
        = 1 - \frac{\binom{48}{5}}{\binom{52}{5}}.
    \]

  \item[(c)] Let $E_c$ be the event that the first two cards are a pair of the
    same rank.  $E_c$ occurs if the second card has the same rank as the first
    card. Therefore, one has
    \[ \pr\left(E_c\right) = \frac{3}{51} = \frac{1}{17}.\]

  \item[(d)] Let $E_d$ be the event that the first five cards are all diamonds. Then,
    \[
      \pr\left(E_d\right)
        = \frac{\binom{13}{5} \cdot \binom{39}{0}}{\binom{52}{5}}
        = \frac{\binom{13}{5}}{\binom{52}{5}}
        = \frac{13 \cdot 12 \cdot 11 \cdot 10 \cdot 9}{52 \cdot 51 \cdot 50 \cdot 49 \cdot 18}
        = \frac{33}{66,640}.
    \]

  \item[(e)] Let $E_e$ be the event that the first five cards form a full house
    (three of one rank and two of another rank). Let us think of event $E_e$ as
    drawing 5 cards from a deck. Two of them have to be from different rank, so
    there are $\binom{13}{2}$ possibilities two choose two different ranks.
    Altogether we have to draw three from one of the picked rank and two of the
    other. Since it doesn't matter which rank appears three and which two times,
    one has
    \[
      \pr\left(E_e\right)
        = \frac{\binom{13}{2} \cdot \binom{4}{3} \cdot \binom{4}{2} \cdot 2}{\binom{52}{5}}
        = \frac{78 \cdot 4 \cdot 6 \cdot 2}{2,598,960}
        = \frac{6}{4,165}.
    \]
\end{enumerate}
