\textbf{Exercise 1.12}: Monty Hall problem: \\
Intuitive solution: Assume that the contestant made her first choice. Since the
contestant had initially no further information , the probability that it is
behind one of the other curtains is $\frac{2}{3}$. Monty then opens one of the
other curtains to show a goat. Consequently the probability that the car is behind
the revealed curtain is 0. Since the probability that the car is behind one
of the non-chosen curtains is $\frac{2}{3}$, the probability that the car is behind
the not-chosen and not-revealed curtain is now $\frac{2}{3}$. Therefore, the
contestant should switch curtains.

Let's analyse the given problem formally. First, we define the sample space $\Omega$.
Let $i \in \{ 1, 2, 3 \}$ and $C_i$ the event that the car is behind curtain $i$.
Then $\Omega = \{ C_1, C_2, C_3 \}$. Furthermore, let $O_i$ be the event that Monty
opens curtain $i$. Without loss of generality assume that the contestant chooses
curtain 1 and that Monty opens curtain 2. Now we are interested in the probabilities
$\pr\left(C_1 | O_2\right)$ and $\pr\left(C_2 | O_2\right)$. Since $C_1, C_2$ and
$C_3$ are mutually disjoint sets such that $\cup_{i=1}^3 C_i = \Omega$, Bayes' Law
applies and one has
\[
  \pr\left(C_1 | O_2\right)
    = \frac {\pr\left(O_2 | C_1\right) \cdot \pr\left(C_1 \right)} {\sum_{i=1}^3 \pr\left(O_2 | C_i\right) \cdot \pr\left(C_i\right)}
    = \frac {\frac{1}{2} \cdot \frac{1}{3}}{\frac{1}{2} \cdot \frac{1}{3} + 0 \cdot \frac{1}{3} + 1 \cdot \frac{1}{3}}
    = \frac{1}{3},
\]
and
\[
  \pr\left(C_3 | O_2\right)
    = \frac {\pr\left(O_2 | C_3\right) \cdot \pr\left(C_3 \right)}{\sum_{i=1}^3 \pr\left(O_2 | C_i\right) \cdot \pr\left(C_i\right)}
    = \frac {1 \cdot \frac{1}{3}}{\frac{1}{2} \cdot \frac{1}{3} + 0 \cdot \frac{1}{3} + 1 \cdot \frac{1}{3}}
    = \frac{2}{3}.
\]
Since $\pr\left(C_3 | O_2\right) > \pr\left(C_1 | O_2\right)$, the contestant
should switch the curtains.
\\[0.5cm]
