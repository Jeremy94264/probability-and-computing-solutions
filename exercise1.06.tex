\textbf{Exercise 1.6}: Balls-and-bin game: We start with one black ball and one
white ball in a bin. We repeatedly choose one ball from the bin uniformly at
random, and then put the ball back in the bin with another ball of the same color.
We repeat until there are $n$ balls in the bin. \\[0,2cm]
Let $X$ be the number of white balls and $k \in \mathbb{N}$, so that $0 \leq k
\leq n-2$. We want to show that,
\[ \pr(X=k+1) = \frac{1}{n-1}. \]
To model this $(n-2)$-stage balls-and-bin game, we describe the drawing of a white
ball as the event 1 and and a black ball as the event 0 respectively. Then our
sample space $\Omega$ can be defined as $\Omega := \Omega_1 \times ... \times
\Omega_{n-2}$, where $\Omega_j = \{ 0, 1\}$ and $j = 1,...,n-2$. Now one concrete
balls-and-bin game can be described as an elementary event $\omega =
(a_1,...,a_{n-2}) \in \Omega$, where $a_j \in \Omega_j$ specifies the $j$-th ball
that we draw. Then
\begin{align*}
  \pr(\omega)
    &= \pr(a_{n-2}| a_1,...,a_{n-3}) \cdot \pr(a_1,...,a_{n-3}) \\
    &= \pr(a_1) \cdot \pr(a_2| a_1) \cdot \pr(a_3| a_1,a_2) \cdot ... \cdot \pr(a_{n-2}| a_1,...,a_{n-3}).
\end{align*}
Let $p_j(1)$ be the probability that the $j$-th ball that we draw is
a white ball and $p_j(0)$ the probability that the $j$-th ball is a black one. \\
Initially there is one white and one black ball in the bin. Therefore,
\[ p_1(1) = \frac{1}{2} = p_1(0). \]
Suppose we have drawn $j-1$ balls and there has been $w$ white and $j-1-w$ black
ones among them. Consequently there are $1 + w$ white balls and $1 + j - 1 - w$
black balls in the bin before the draw the $j$-th ball. Let
$\left(a_1,...,a_{j-1}\right)$ be a tuple with $w$ 1's and $(j-1-w)$ 0's, that is
$\sum_{i=1}^{j-1}a_i = w$. Then
\begin{align*}
    p_j(1|a_1,...,a_{j-1}) = \frac{1 + w}{1 + 1 + (j-1)}, \\
    p_j(0|a_1,...,a_{j-1}) = \frac{1+j-1-w}{1 + 1 + (j-1)}.
\end{align*}
Let $\omega = (a_1, ..., a_{n-2}) \in \Omega$, so that $\sum_{j=1}^{n-2}a_j = k$
and $0 \leq k \leq n-2$. Then
\[
  \pr(\omega) = \frac {\prod_{j=0}^{k-1}(1 + j) \prod_{j=0}^{(n-2)-k-1}(1 + j)}
                      {\prod_{j=0}^{n-3}(1+1+j)}.
\]
Since the event $X=k+1$ consists of all tuples $(a_1,...,a_{n-2})$ with
$\sum_{j=1}^{n-2}a_j = k$ and since each of these tuples has the above probability,
\begin{align*}
  \pr(X=k+1)
    &= \binom{n-2}{k} \cdot \frac {\prod_{j=0}^{k-1}(1 + j) \prod_{j=0}^{(n-2)-k-1}(1 + j)}{\prod_{j=0}^{n-3}(1+1+j)} \\
    &= \frac{(n-2)!}{k! (n-2-k)!} \cdot \frac{k! \cdot (n-2-k)!}{(n-1)!} \\
    &= \frac{1}{n-1}.
\end{align*}
Hence the number of white balls is equally likely to be any number between 1 and
$n-1$.
