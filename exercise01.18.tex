\textbf{Exercise 1.18}: Suppose we are given an integer $z \in \{ 0, ..., n-1 \}$
and want to output a value that equals $F(z)$ with probability at least
$\frac{1}{2}$. \\
We use a randomized algorithm that chooses a random number $x \in \{ 0, ..., n-1 \}$.
It then computes $y = (z - x) \text{ mod }n$ and afterwards uses the lookup table
to output the value $(F(x) + F(y)) \text{ mod }m$. \\
If the adversary has not changed the values of $x$ and $y$ we can be sure that
our algorithm outputs the correct value. The probability that the adversary has
changed neither $x$ nor $y$ is $\frac{4}{5} \cdot \frac{4}{5} = \frac{16}{25}$.
Since $\frac{16}{25} \geq \frac{1}{2}$, our algorithm outputs a value that equals
F(z) with probability at least $\frac{1}{2}$.

Suppose that we are allowed to repeat our initial algorithm three times. Let $i
\in \{ 1, 2, 3 \}$ and let $a_i$ be the output of the $i$th application of the
algorithm. We use the following strategy: If at least two outcomes match, we output
the corresponding value. Otherwise the output $a_1$.  \\
 Let $C$ be the event that we output the correct answer. To assess $\pr(C)$, we
consider the probability of the complementary event. The probability that we
output the wrong value is greater than zero if and only if there are at least
two wrong outputs among the three applications of our algorithm. The probability
that there are at least two wrong outputs is $\left(\frac{9}{25}\right)^2$.
Therefore
\[ \pr(C) = 1 - \pr(\bar{C})\geq 1 - \left(\frac{9}{25}\right)^2 = \frac{544}{625}.\]
Hence,the probability $\pr(C)$ that our enhanced algorithm returns the correct
answer is at least $0.8704$.
\\[0.5cm]
