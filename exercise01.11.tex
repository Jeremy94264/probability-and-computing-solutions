\paragraph{Exercise 1.11} Assume we are trying to send a single bit, either a 0
 or a 1. When we transmit the bit, it goes through a series of $n$ relays
 before it arrives. Each relay flips the bit independently with probability $p$.
 \begin{enumerate}
   \item[(a)] We are interested in the probabilty to receive the correct bit.
    If an odd number of relays flips the bit an incorrect bit arrives and if
    an even number of relays flips the bit the correct bit arrives. Let $k \in
    \mathbb{N}$ and let $E_k$ be the event that exactly $2 \cdot k$ relays flip
    the bit. Then
    \[ \pr\left(\cup_{k=0}^{\left\lfloor n/2 \right\rfloor} E_k \right) \]
    is the probability to receive the correct bit.
    Since $E_0,...,E_{\left\lfloor n/2 \right\rfloor}$ are mutually
    disjoint events,
    \[
      \pr\left(\cup_{k=0}^{\left\lfloor n/2 \right\rfloor} E_k \right)
      = \sum_{k=0}^{\left\lfloor n/2 \right\rfloor} \pr\left( E_k \right).
    \]
    Let $k \in \mathbb{N}$. There are $\binom{n}{2k}$ possibilities to choose $2k$
    relays among the $n$ relays that flip the bit. The probability that $2k$
    specified relays flip the bit, is $p^{2k}(1-p)^{n-2k}$. Therefore
    \[ \pr\left(E_k\right) = \binom{n}{2k} p^{2k}(1-p)^{n-2k}, \]
    and
    \[
      \pr\left(\cup_{k=0}^{\left\lfloor n/2 \right\rfloor} E_k \right)
      = \sum_{k=0}^{\left\lfloor n/2 \right\rfloor} \pr\left( E_k \right)
      = \sum_{k=0}^{\left\lfloor n/2 \right\rfloor} \binom{n}{2k} p^{2k}(1-p)^{n-2k}.
    \]

   \item[(b)] Let us say the relay has bias $q$, there $q \in \mathbb{R}$,
   $q \in [-1, 1]$ if the probability it flips the bit is $(1 - q)/2$. We want
   to show that sending a bit through two relays with bias $q_1$ and $q_2$ is
   equivalent to sending a bit through a single relay with bias $q_1q_2$. \\
   Let $X \in $ \{correct, incorrect\} indicate whether we receive the correct bit
   after sending it through two relays with bias $q_1$ and $q_2$. Similarly, let
   $Y \in $ \{correct, incorrect\} indicate whether we receive the correct bit
   after sending it through one relays with bias $q_1q_2$. According to the
   definition of the bias of a relay, one has
   \[ \pr(Y = \text{incorrect}) = \frac{1 - q_1q_2}{2}. \]
   Let's consider the event $\{ X = \text{incorrect} \}$. This event occurs if and
   only if exactly one of the two relays flips the bit. Therefore
   \begin{align*}
     \pr\left( X = \text{incorrect} \right)
      &= \frac{1-q_1}{2} \cdot \left(1 - \frac{1-q_2}{2}\right) +
         \frac{1-q_2}{2} \cdot \left(1 - \frac{1-q_1}{2}\right) \\
      &= \frac{1-q_1}{2} - \frac{(1-q_1)(1-q_2)}{4} +
         \frac{1-q_2}{2} - \frac{(1-q_2)(1-q_1)}{4} \\
      &= \frac{1}{4} \left( 2(1-q_1) - (1-q_1)(1-q_2) + 2(1-q_2) - (1-q_2)(1-q_1) \right) \\
      &= \frac{1}{2} \left((1-q_1) - (1-q_1)(1-q_2) + (1-q_2) \right) \\
      &= \frac{1}{2} \left(1 - q_1 - (1 - q_2 - q_1 + q_1q_2) + 1 - q_2 \right) \\
      &= \frac{1}{2} \left(1 - q_1 - 1 + q_2 + q_1 - q_1q_2 + 1 - q_2 \right) \\
      &= \frac{1}{2} \left(1 - q_1q_2 \right). \\
   \end{align*}
   Hence, $\pr\left( X = \text{incorrect} \right) = \pr\left( Y = \text{incorrect}
   \right)$, and consequently sending a bit through two relays with bias $q_1$ and
   $q_2$ is equivalent to sending a bit through a single relay with bias $q_1q_2$.

   \item[(c)] Theorem: The probability you receive the correct bit when it passes
    through $n$ relays is
    \[ \frac{1 + (1 - 2p)^n}{2}. \]
    \textit{Proof}. We show the above theorem by induction over $n \in \mathbb{N}^+$.
    As the base case, assume that $n = 1$. The probability you receive the correct
    bit equals the probability that the only relay doesn't flip, that is $(1 - p)$.
    Since
    \[ \frac{1 + (1 - 2p)^1}{2} = \frac{2 - 2p}{2} = (1 - p), \]
    the theorem applies for the base case. \\
    Let $n \in \mathbb{N}^+$, so that the probability you receive the correct bit
    when it passes through $n$ relays is $\frac{1 + (1 - 2p)^n}{2}$ (induction
    hypothesis, IH). We have to show that the theorem also applies for $n+1$
    relays. Let $E_1$ be the event that the bit is correct after it passes through
    $n$ relays and the $(n+1)$th relay doesn't flip the bit. Let $E_2$ be the
    event that the bit is incorrect after it passes through $n$ relays and the
    $(n+1)$th relay flips the bit. Let $E$ be the event that you receive the correct
    bit when it passes through $n + 1$ relays. Then $E = E_1 \cup E_2$. Since
    $E_1$ and $E_2$ are mutually disjoint, one has
    \begin{align*}
      \pr(E)
      &= \pr(E_1) + \pr(E_2) \\
      &=_{\text{IH}}  \left( \frac{1 + (1 - 2p)^n}{2} \cdot (1 - p) \right) +
        \left( \left( 1 - \frac{1 + (1 - 2p)^n}{2} \right) \cdot p \right) \\
      &= \frac{1}{2} \left( (1-p) + (1-p)(1 - 2p)^n \right) +
        \frac{1}{2} \left( 2p - (p + p(1 - 2p)^n \right) \\
      &= \frac{1}{2} \left( 1 - p + (1-p)(1 - 2p)^n + 2p - p - p(1 - 2p)^n \right) \\
      &= \frac{1}{2} \left( 1  + (1 - 2p)^n(1 - p - p) \right) \\
      &= \frac{1  + (1 - 2p)^{n+1}}{2}.
    \end{align*}
    Hence, the probability you receive the correct bit when it passes through $n$
    relays is $\frac{1 + (1 - 2p)^n}{2}$.
 \end{enumerate}
